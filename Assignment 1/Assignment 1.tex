\documentclass[12pt,letterpaper]{article}
\usepackage[margin=1in]{geometry}
\usepackage{setspace} \doublespacing
\usepackage{hanging}

\begin{document}
\begin{center}
	\textbf{\underline{Memorandum}}
\end{center}
\begin{tabular}{ll}
	\textbf{To:}	& Joe Radocchia\\
	\textbf{From:}	& Raeed Hassan\\
	\textbf{Date:}	& \today\\
	\textbf{Re:}	& \textbf{McMaster Soils Case}\\
\end{tabular}
\\\par
In Canadian tort law, as reaffirmed in \textit{Rankin's Garage \& Sales v. J.J.}, the principles for the Tort of Negligence are established based on the neighbour principle established by Donoghue v. Stevenson (Radocchia). The plaintiff must prove there is a relationship of proximity between the parties that should establish a \textit{prima facie} duty of care and must provide a sufficient factual basis that the harm caused was a reasonably foreseeable consequence of the actions by a defendant in a proximate relationship. This is the basis that will used to determine liability for the \textit{McMaster Soils Case}.

\par
\textbf{\underline{Relevant Parties}}\\
\indent
The relevant parties to this case are McMaster University (Mac), the architect hired by Mac, Saul Goodman (Saul), the engineering firm contracted by Saul, Fring Engineering Services (FES), and Fring's employee engineer, Kim Wexler (Kim). If Mac sue the other parties relevant to this scenario, they must prove a relationship of proximity with the parties. In this scenario, Mac has directly established a relationship with Saul, who was hired by Mac, and Kim, who provided Mac with a "soils report". It is unlikely Mac will be able to prove \textit{prima facie} duty of care from FES, however they may attempt to hold FES vicariously liability if it's employee, Kim, can be found primarily liable.

\par
\textbf{\underline{\textit{Prima Facie} Duty of Care}}\\
\indent
The first step of the harmed party, Mac, is to establish that there must a \textit{prima facie} duty of care from the parties they attempt to claim are primarily liable. In this fact scenario, Mac hired Saul as an architect to design a new library and it can be reasonably assumed that in a working relationship, both parties have a \textit{prima facie} duty of care for the other. The other party that Mac had a direct relationship with was Kim, who directly delivered a “soils report” to Mac while working as an engineer for a firm hired by Sam. As an engineer and expert, it could be argued that Kim had a duty of care to Mac when delivering this report. As Mac has not directly interacted with FES, Kim's employer, Mac will not be able establish a \textit{prima facie} duty of care from FES.\par
In the given fact scenario, it is likely that Mac will be able to establish a prime facie duty of care for Saul and Kim but will not be establish a \textit{prima facie} duty of care for FES.

\par
\textbf{\underline{Foreseeable Consequences}}\\
\indent
Since it is believed that Mac will be able to establish a \textit{prima facie} duty of care with multiple parties in a potential case, we can continue to discuss if there are sufficient facts to establish that the harm required by Mac was a reasonably foreseeable. To establish this, we will refer to second to fourth parts of the four part objective test to succeed a claim in negligence. In this test, Mac will be the plaintiff, and Saul and Kim will be defendants.\par
We must ask ``[d]id the defendant breach the required standard of care expected of a reasonable person?" (Radocchia). In the given scenario, it can be argued the expected standard of care from the defendants is to provide the relevant information to determine if the library can be built safely. Saul failed this expectation when he failed to relay the expert advice he received from Kim. Kim failed this expectation when he failed to raise concern as an engineer and expert when Saul attempted to go ahead with the design of the library without doing sufficient soil testing. As both defendants are believed to have breached the required standard of care, we can proceed to the next test.\par
We must determine if the victim (Mac) suffered any harm. In this scenario, Mac had to pay additional expenses for extensive remedial foundation work to correct the settlement problems, suffering financial harm. As Mac is believed to have suffered harm, we can proceed to the final test.\par
As we believe all previous tests have been passed, we can proceed to the final test of determining if the defendant's actions cause the harm to the victim, or if their actions can be disconnected from the harm. In this case, the financial harm inflicted on Mac is a result of the actions of Saul and Kim, as Mac was relying on the expert advice of Saul and Kim for the safe and stable construction of the new library and the defendants in this case failed to provide them with adequate information for Mac to work with.

\par
\textbf{\underline{Likely Court Proceedings and Outcome}}\\
\indent
If this matter reaches courts, it is likely that the court and judge will apply the same established four part objective test to determine negligence. The court would likely award damages to Mac that will at least cover the cost for extensive remedial foundation work that was performed on the new library. Saul and Kim would be found primarily liable for the Tort of Negligence and find FES vicariously liable for the actions of their employee Kim. 

\clearpage 
\noindent
\textbf{\underline{References}}\\
\hangpara{1em}{2}
Radocchia, Joe. \textit{ENGINEER 4A03 POWER POINT SLIDES for October 17, 2022}. ENGINEER 4A03, 17 Oct. 2022. McMaster University, Hamilton. PowerPoint Presentation. 
\end{document}